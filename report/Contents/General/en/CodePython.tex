%%%%%%%%%%%%%%%%%%%%%%%%
%
% $Autor: Wings $
% $Datum: 2020-07-24 09:05:07Z $
% $Pfad: GDV/Vortraege/latex - Report/Contents/Python.tex $
% $Version: 4732 $
%
%%%%%%%%%%%%%%%%%%%%%%%%

\chapter{Representation of Python Programs}


It is possible to integrate a part into the document, see \ref{Code:Python:File:HelloWorld}. This is the most elegant method and is also preferable. Individual lines and line ranges can also be selected in the list of options. If a file is integrated, it is always as up-to-date as the document.


\begin{code}
  \caption[\glqq Hello World\grqq{} in Python -- Variant 1]{The program ``Hello World'' in Python for microcontroller boards is inserted from the file \FILE{Blink.py}.}\label{Code:Python:File:HelloWorld}    

  \pythonexternal{}{../../Code/HelloWorld/Blink.py}    
    
\end{code}    

It may also be useful to integrate program lines: \PYTHON{redLED = pyb.LED(1)}.


\bigskip

Attention: The integration of programs with the help of images is pointless.